Puji syukur ke hadirat Allah SWT yang telah melimpahkan rahmat, karunia, dan bimbingan-Nya yang tak terhingga sehingga penulis dapat menyelesaikan skripsi dengan judul "Optimalisasi\textit{ Knowledge Extraction} dan \textit{Knowledge Retrieval} pada RAG untuk \textit{Chatbot} Kesehatan Mental".
Laporan skripsi ini disusun untuk memenuhi salah satu syarat dalam memperoleh gelar Sarjana Teknik (S.T.) pada Program Studi S1 Teknologi Informasi Fakultas Teknik Universitas Gadjah Mada Yogyakarta.


Perjalanan penelitian dan penyusunan skripsi ini tidak akan terwujud tanpa dukungan, bimbingan, dan semangat dari berbagai pihak. Oleh karena itu, penulis ingin menyampaikan ucapan terima kasih yang tulus kepada:

\begin{enumerate}
	\item Prof. Ir. Hanung Adi Nugroho, S.T., M.E., Ph.D., IPM., SMIEEE. selaku Ketua Departemen Teknik Elektro dan Teknologi Informasi Fakultas Teknik Universitas Gadjah Mada

	\item  Ir. Lesnanto Multa Putranto, S.T., M.Eng., Ph.D., IPM., SMIEEE. selaku Sekretaris Departemen Teknik Elektro dan Teknologi Informasi Fakultas Teknik Universitas Gadjah Mada

	\item Dr. Bimo Sunarfri Hantono, S.T., M.Eng. selaku dosen pembimbing pertama, yang dengan penuh kesabaran dan ketulusan selalu memberikan arahan, bimbingan, serta dukungan dalam proses penelitian hingga penyusunan skripsi ini.

	\item Dr. Guntur Dharma Putra, S.T., M.Sc. selaku dosen pembimbing kedua, yang dengan penuh perhatian dan kesungguhan senantiasa memberikan masukan, dorongan, serta ilmu yang sangat berharga.

	\item Orang tua dan adik, atas doa, kasih sayang, serta dukungan yang tiada henti yang menjadi sumber semangat terbesar dalam setiap langkah.

	\item Teman-teman seperjuangan, sahabat, serta semua pihak yang tidak dapat penulis sebutkan satu per satu, yang telah memberikan semangat, bantuan, dan kebaikan sepanjang perjalanan ini.


\end{enumerate}

Penulis menyadari bahwa skripsi ini masih memiliki keterbatasan dan jauh dari kata sempurna. Oleh karena itu, segala bentuk kritik dan saran yang membangun akan penulis terima dengan tangan terbuka.
Akhir kata, semoga tulisan ini dapat memberikan manfaat, menambah wawasan, serta menjadi pijakan bagi penelitian selanjutnya.

\begin{flushright}
	\begin{tabular}{c}
		Yogyakarta, 17 September 2025 \\
		\vspace{1cm}                  \\
		Ega Rizky Setiawan
	\end{tabular}
\end{flushright}