\chapter{Pendahuluan}

\section{Latar Belakang}
Kesehatan mental merupakan komponen fundamental bagi kesejahteraan individu dan masyarakat secara keseluruhan.
Isu kesehatan mental global menunjukkan tren yang mengkhawatirkan, ditandai dengan peningkatan prevalensi berbagai gangguan mental seperti depresi, kecemasan, dan stres di berbagai negara.
Sebagai contoh, penelitian yang dilakukan oleh Institute for Health Metrics and Evaluation (IHME) memperkirakan bahwa sekitar 970 juta orang, atau 12\% dari populasi global, mengalami gangguan mental pada 2019 \cite{GBD2019MentalDisorders}.
Di Indonesia, situasinya juga tidak kalah mengkhawatirkan. Khususnya pada populasi remaja, hasil temuan penelitian Indonesia - National Adolescence Mental Health Survey (I-NAMHS) yang dirilis pada tahun 2022 menyajikan gambaran yang lebih detail:
satu dari tiga remaja Indonesia (34,9\%), setara dengan 15,5 juta jiwa, memiliki setidaknya satu masalah kesehatan mental dalam 12 bulan terakhir.
Lebih lanjut, satu dari dua puluh remaja (5,5\%), atau sekitar 2,45 juta jiwa, terdiagnosis mengalami gangguan mental dalam periode yang sama, dengan gangguan kecemasan sebagai jenis yang paling umum ditemui \cite{INAMHS2022}.
Dampak dari gangguan kesehatan mental ini sangat luas, tidak hanya memengaruhi kualitas hidup, produktivitas, dan interaksi sosial individu, tetapi juga memberikan beban signifikan pada sistem layanan kesehatan dan perekonomian.
Oleh karena itu, investasi dalam upaya peningkatan kesehatan mental sangat krusial, karena berpotensi meningkatkan kualitas kesehatan masyarakat secara umum, yang pada gilirannya akan mendukung peningkatan fungsi sosial, produktivitas, dan pertumbuhan ekonomi.

Sayangnya, akses terhadap layanan kesehatan mental yang berkualitas masih menjadi tantangan besar.
Faktor-faktor seperti stigma sosial yang kuat, keterbatasan jumlah tenaga profesional kesehatan mental yang terdistribusi tidak merata, serta biaya layanan yang sering kali tidak terjangkau oleh sebagian besar masyarakat, menciptakan kesenjangan besar antara kebutuhan dan ketersediaan layanan.
Sebanyak 71\% dari penderita psikosis tidak mendapatkan penanganan kesehatan mental \cite{WHO2022MentalHealth}. Data I-NAMHS 2022 semakin memperjelas jurang ini:
hanya 2,6\% dari remaja dengan masalah kesehatan mental yang pernah mengakses layanan yang menyediakan dukungan atau konseling untuk masalah emosi dan perilaku dalam 12 bulan terakhir.
Lebih lanjut, secara keseluruhan, hanya satu dari lima puluh remaja (2\%) yang pernah menggunakan layanan dalam 12 bulan terakhir, dan dua pertiga dari mereka (66,5\%)
hanya pernah menggunakan layanan sebanyak satu kali. Ironisnya, ketika ditanyakan mengenai penyedia layanan yang paling banyak diakses oleh mereka yang mencari bantuan, hampir dua perlima (38,2\%)
pengasuh utama menjawab petugas sekolah, yang mengindikasikan peran penting namun mungkin belum optimal dari sistem pendidikan dalam deteksi dini dan rujukan. Fenomena ini juga diperkuat oleh temuan bahwa
hanya 4,3\% dari pengasuh utama yang menyatakan bahwa remaja mereka membutuhkan bantuan untuk masalah emosi dan perilaku, meskipun data menunjukkan 34,9\% remaja mengalami masalah kesehatan mental.
Dari pengasuh utama yang menyadari kebutuhan bantuan, lebih dari dua perlima (43,8\%) melaporkan tidak mencari bantuan karena lebih memilih untuk menangani sendiri masalah remaja tersebut atau dengan dukungan
dari keluarga dan teman-teman, yang mungkin juga mencerminkan stigma atau ketidakpercayaan terhadap layanan formal. Kondisi ini mendesak perlunya inovasi untuk memperluas jangkauan dan meningkatkan kualitas dukungan kesehatan mental.
Pandemi COVID-19 juga turut memperburuk situasi, di mana 4,6\% remaja melaporkan sering merasa lebih cemas, lebih depresi, lebih kesepian, atau lebih sulit berkonsentrasi dari biasanya selama periode tersebut \cite{INAMHS2022}.

Teknologi kecerdasan buatan atau \textit{Artificial Intelligence} (AI) telah berkembang secara masif selama beberapa dekade terakhir mulai menggeser perubahan aktivitas manusia.
AI mengubah proses-proses tradisional yang kaku dan membutuhkan campur tangan manusia secara intens menjadi proses modern yang luwes dan minim campur tangan manusia.
Teknologi ini sudah banyak mengintervensi kegiatan sehari-hari di berbagai sektor industri sperti \textit{online shopping}, \textit{online banking}, dan \textit{mental helth care} \cite{MultiIndustryAIChatbot}.
Salah satu aplikasi teknologi AI dalam kesehatan mental yang paling banyak dikembangkan adalah \textit{chatbot} atau agen percakapan.
\textit{Chatbot} kesehatan mental menawarkan potensi solusi yang menjanjikan dengan memberikan dukungan emosional awal dan informasi psikoedukasi yang mudah diakses kapan saja dan di mana saja melalui perangkat digital.
Keunggulan tersebut diharapkan dapat membantu mengatasi beberapa hambatan penderita mental disorder dalam mencari bantuan, terutama terkait dengan stigma dan aksesibilitas geografis maupun finansial \cite{CBTWoebotTest}.
Salah satu penerapan \textit{chatbot} pada bidang kesehatan mental adalah Lintang \textit{Chatbot} yang dikembangkan oleh Health Promoting University (HPU) Universitas Gadjah Mada (UGM).
\textit{Chatbot} ini berbasis pada aturan (\textit{rule based}) yang dirancang untuk memberikan pertolongan pertama psikologis dan informasi kesehatan mental.

Meskipun menawarkan potensi yang menjanjikan, \textit{chatbot} kesehatan mental yang ada saat ini, utamanya yang berbasis pada aturan atau model generatif sederhana, masih memiliki keterbatasan.
Sering kali respons yang diberikan terasa kaku, repetitif, dan gagal memahami nuansa dalam ekspresi emosi atau pertanyaan pengguna.
Tantangan utama terletak pada kemampuan \textit{chatbot} dalam mengakses, memproses, dan menyajikan informasi yang tidak hanya akurat dan relevan, tetapi juga kontekstual dan empatik.
Pengguna sering kali membutuhkan informasi yang sangat spesifik mengenai kondisi mereka, opsi perawatan, atau fasilitas yang tersedia.
Informasi ini sering kali tidak tersedia oleh \textit{chatbot} pada umumnya karena keterbatasan terhadap data yang selalu berubah dan berkembang setiap waktu.
Keterbatasan akses tersebut dapat membuat model generatif mengalami "halusinasi", yaitu model menciptakan informasi yang terdengar meyakinkan namun tidak faktual atau tidak bersumber dari basis pengetahuannya.
Lebih lanjut, memastikan bahwa basis pengetahuan \textit{chatbot} selalu terbarui, berbasis bukti ilmiah (\textit{evidence-based}), dan bebas dari misinformasi merupakan hal yang krusial, mengingat sensitivitas isu kesehatan mental.

Untuk menjawab tantangan tersebut, salah satu pendekatan yang telah dikembangkan adalah \textit{Retrieval-Augmented Generation} (RAG) \cite{Lewis2021RAGKnowledgeIntensiveNLP}.
Pendekatan ini memanfaatkan kekuatan model bahasa generatif seperti \textit{Large Language Model} (LLM).
LLM merupakan model kecerdasan buatan yang telah dilatih dengan data teks dalam jumlah sangat besar, memungkinkannya untuk menghasilkan respons yang sangat luwes dan alami.
RAG kemudian menggabungkan kemampuan generatif ini dengan informasi relevan yang diambil dari basis pengetahuan.
Pendekatan tersebut diharapkan \textit{chatbot} dapat memberikan jawaban yang lebih faktual, mengurangi halusinasi, dan menyajikan informasi yang lebih spesifik dan terkini.
Kualitas RAG dipengaruhi oleh setidaknya 2 komponen yaitu efektivitas mekanisme \textit{retrieval} dan kualitas basis pengetahuan yang digunakan.

Dalam domain yang kaya akan informasi terstruktur dan relasi kompleks seperti kesehatan mental, penggunaan \textit{Knowledge Graph} (KG) sebagai basis pengetahuan eksternal menawarkan keunggulan signifikan dibandingkan sumber data teks tidak terstruktur.
\textit{Knowledge Graph} merepresentasikan pengetahuan sebagai jaringan entitas (misalnya, jenis gangguan, gejala, terapi, obat) dan relasi semantik di antara mereka
(misalnya, "Depresi Mayor" memiliki gejala "Kehilangan Minat", "Terapi Kognitif Perilaku" digunakan untuk mengobati "Gangguan Panik") \cite{Chen2020ReviewKnowldgeReasoningOverKnowledgeGraph}.
Representasi terstruktur ini memungkinkan pencarian informasi yang lebih presisi, penarikan kesimpulan (\textit{reasoning}), dan pemahaman konteks yang lebih mendalam.
Integrasi KG ke dalam RAG berpotensi meningkatkan kemampuan \textit{chatbot} kesehatan mental untuk memberikan penjelasan yang kaya, jawaban yang akurat atas pertanyaan-pertanyaan spesifik, dan bahkan membantu pengguna memahami hubungan antara berbagai aspek kondisi mental mereka.

Namun, efektivitas KG dalam arsitektur RAG sangat ditentukan oleh dua proses fundamental yaitu \textit{Knowledge Extraction} (Ekstraksi Pengetahuan) dan \textit{Knowledge Retrieval} (Pengambilan Pengetahuan).
\textit{Knowledge Extraction} adalah proses mengidentifikasi dan mengekstrak informasi terstruktur (entitas, relasi, atribut) dari berbagai sumber data (misalnya, literatur medis, jurnal psikologi, pedoman klinis, forum diskusi terkurasi) untuk membangun atau memperkaya KG \cite{Choi2025KnowledgeGraphConstruction}.
Dalam domain kesehatan mental, proses ini dihadapkan pada tantangan seperti variasi terminologi, sinonim, polisemi, dan kebutuhan krusial akan akurasi serta validitas informasi.
Ekstraksi yang tidak optimal akan menghasilkan KG yang tidak lengkap, tidak akurat, atau bias, yang pada akhirnya akan menurunkan kualitas informasi yang dapat diakses oleh sistem RAG.

Selanjutnya, \textit{Knowledge Retrieval} dari KG untuk diumpankan ke model generatif dalam RAG juga merupakan aspek kritis.
Ketika pengguna berinteraksi dengan \textit{chatbot} menggunakan bahasa alami, sistem harus mampu menerjemahkan pertanyaan atau pernyataan pengguna menjadi kueri yang dapat dipahami oleh KG.
Sistem kemudian harus secara efisien mengambil fragmen pengetahuan (sub-graf) yang paling relevan dari KG untuk mendukung generasi respons.
Proses ini harus mampu menangani ambiguitas bahasa, memahami intensi pengguna secara akurat, dan mengambil informasi yang relevan tanpa membebani model generatif dengan informasi yang berlebihan atau tidak perlu.
Metode \textit{retrieval} yang suboptimal akan menghasilkan respons yang kurang relevan, tidak lengkap, atau bahkan menyesatkan \cite{Edge2025LocalGlobalGraphRAG}.

Meskipun potensi RAG berbasis KG untuk \textit{chatbot} kesehatan mental sangat besar, penelitian yang secara spesifik berfokus pada optimalisasi proses \textit{Knowledge Extraction} dan \textit{Knowledge Retrieval} dalam konteks ini masih relatif terbatas.
Kebanyakan penelitian lebih terfokus pada pembangunan KG secara umum atau aplikasi RAG dengan sumber data teks bebas.
Padahal, optimalisasi kedua proses ini, yang disesuaikan dengan karakteristik unik dan tuntutan domain kesehatan mental seperti kebutuhan akan akurasi tinggi, empati, dan penyampaian informasi yang mudah dicerna sangat esensial untuk memaksimalkan efektivitas dan keamanan \textit{chatbot}.

Berdasarkan latar belakang di atas, penelitian ini diusulkan untuk menginvestigasi dan mengembangkan metode optimalisasi untuk proses \textit{Knowledge Extraction} dan \textit{Knowledge Retrieval} dalam arsitektur RAG yang memanfaatkan \textit{Knowledge Graph} sebagai basis pengetahuan eksternal,
dengan studi kasus pada pengembangan \textit{chatbot} kesehatan mental. Penelitian ini diharapkan dapat memberikan kontribusi pada pengembangan \textit{chatbot} kesehatan mental yang lebih cerdas, andal, dan bermanfaat,
serta menawarkan panduan praktis bagi pengembangan sistem serupa di masa depan, sehingga dapat memperluas aksesibilitas dan kualitas dukungan kesehatan mental berbasis teknologi.


% \begin{enumerate}
% 	\item pentingnya kesehatan mental ??
% 	\item isu kesehatam mental global/indo like menurut WHO/kemenkes ..... OK
% 	\item kendala akses ke psikolog butuh biaya /stigma malu /waktu terbatas OK
% 	\item peran AI dalam menangani kasus kesehatan mental - chatbot OK
% 	\item potensi AI dalam memecahkan masalah kesehatan mental OK
% 	\item chatbot ugm lintang yg masih rule based
% 	\item tantangan chatbot (bahasa yang tidak judging, empati, informasi mutakhir, punya evidence, tidak menyesatkan)

% 	\item RAG sebagai alternatif mengatasi tantangan tsb
% 	\item kinerja RAG dipengaruhi oleh konwledge domainnya aka knowledge extract dan Retrieval
% 	\item Knowledge Graph sebagai sebagai model/desain/skema/archi untuk memodelkan knowledge
% 	\item integrasi KG pada RAG menjawab tantangan evidence

% 	\item Tantangan integrasi KG pada RAG
% 	\item knowledge Extraction
% 	\item knowledge Retrieval
% 	\item dengan berkembangnya LLM yang masiv dpt membantu 2 proses ituhh
% 	\item oleh karena ituu....
% \end{enumerate}

\section{Rumusan Masalah}
Berdasarkan uraian pada latar belakang, terlihat adanya potensi besar pemanfaatan \textit{chatbot} sebagai alat bantu layanan kesehatan mental yang aksesibel.
Namun, potensi ini dihadapkan pada tantangan fundamental yaitu memastikan bahwa informasi dan respons yang diberikan tidak hanya simpatik, tetapi juga akurat, dapat diandalkan, dan berbasis bukti ilmiah.
Kegagalan dalam memenuhi standar ini dapat berisiko memberikan informasi yang salah atau bahkan berbahaya bagi pengguna.
Pengembangan arsitektur \textit{chatbot} canggih yang memanfaatkan teknologi seperti LLM dan KG menawarkan solusi yang menjanjikan, namun implementasinya masih menghadapi berbagai kendala teknis yang signifikan.
Oleh karena itu, penelitian ini merumuskan beberapa permasalahan utama yang perlu diatasi untuk mewujudkan \textit{chatbot} kesehatan mental yang aman dan efektif.
\begin{enumerate}
	\item Belum adanya rancangan arsitektur \textit{chatbot} layanan kesehatan mental yang teruji untuk menjamin konsistensi dan akurasi respons berbasis bukti, sehingga berisiko menimbulkan misinformasi yang dapat membahayakan pengguna dalam konteks kesehatan mental yang sensitif.
	\item Adanya tantangan dalam merancang proses \textit{Knowledge Extraction} menggunakan LLM yang mampu membangun KG yang terstruktur secara efektif untuk kebutuhan \textit{Knowledge Retrieval}. Ketidakefektifan proses \textit{Knowledge Extraction} dapat menghasilkan KG yang relasinya sulit diinterpretasi oleh algoritma \textit{retrieval}, sehingga informasi penting gagal ditemukan saat pengguna bertanya.
	\item Ketidakmampuan proses \textit{Knowledge Retrieval} untuk secara konsisten menemukan potongan informasi yang paling relevan dari KG sebagai respons terhadap pertanyaan pengguna, berisiko membuat arsitektur RAG mengambil basis bukti yang salah atau tidak lengkap, sehingga respons yang dihasilkan menjadi tidak akurat dalam menjawab pertanyaan spesifik pengguna.
\end{enumerate}
\section{Tujuan Penelitian}
Berdasarkan rumusan masalah yang telah dipaparkan, penelitian ini memiliki beberapa tujuan spesifik yang dirancang untuk menjawab tantangan tersebut.
Adapun tujuan dari penelitian ini adalah sebagai berikut:
\begin{enumerate}
	\item Merancang dan mengimplementasikan sebuah arsitektur \textit{chatbot} kesehatan mental yang mengintegrasikan KG dan RAG untuk menghasilkan respons berbasis bukti yang konsisten dan akurat.
	\item Mengembangkan metode \textit{Knowledge Extraction} menggunakan LLM yang mampu membangun KG dengan struktur relasi yang efektif untuk mendukung proses \textit{Knowledge Retrieval}.
	\item Merancang dan menerapkan sebuah metode \textit{Knowledge Retrieval} yang mampu secara akurat dan efisien menemukan informasi paling relevan dari KG sebagai respons terhadap pertanyaan spesifik pengguna, guna meningkatkan akurasi sistem RAG.
\end{enumerate}
\section{Batasan Penelitian}
\begin{enumerate}
	\item \textbf{Objek Penelitian}: Penelitian ini akan berfokus pada metode RAG berbasis \textit{Knowledge Graph} yang mencakup
	      Metode optimalisasi \textit{Knowledge Extraction} dari sumber data tekstual kesehatan mental,
	      struktur, kualitas, dan pemanfaatan \textit{Knowledge Graph} kesehatan mental sebagai basis bukti,
	      algoritma dan strategi optimalisasi \textit{Knowledge Retrieval} dari \textit{Knowledge Graph} untuk mendukung sistem RAG, dan
	      kualitas dan efektivitas respons berbasis bukti yang dihasilkan oleh prototipe \textit{chatbot} kesehatan mental yang dikembangkan.
	\item \textbf{Metode Penelitian}: Penelitian ini akan menggunakan pendekatan metode eksperimen dan kuantitatif.
	      Pada tahap pengembangan akan menggunakan pendekatan desain dan rekayasa sistem (\textit{design and engineering}) untuk merancang, membangun, dan mengimplementasikan metode \textit{Knowledge Extraction}, \textit{Knowledge Graph}, dan \textit{Knowledge Retrieval}.
	      Ini melibatkan studi literatur, perancangan algoritma, implementasi perangkat lunak, dan pengujian iteratif.
	      Selanjutnya evaluasi performa komponen sistem dilakukan secara kuantitatif.
	      Performa komponen \textit{retriever} dan jawaban akhir akan dibandingkan degan \textit{golden standard}.
	\item \textbf{Waktu dan Tempat Penelitian}: Penelitian ini akan dilakukan selama periode enam bulan, dimulai pada bulan Februari 2025 hingga Juli 2025. Tempat penelitian adalah di Gedung DTETI FT UGM, dengan menggunakan VS Code untuk pengembangan dan pengujian model.
	\item	\textbf{Populasi dan Sampel}: Populasi dalam penelitian ini adalah seluruh korpus data teks yang relevan dengan domain kesehatan mental seperti artikel ilmiah dan buku teks.
	      Sampel data dipilih dari populasi tersebut yang berupa buku dan artikel ilmiah yang tersedia pada \textit{website} \href{https://ugm.ac.id/}{UGM}, \href{https://cpmh.psikologi.ugm.ac.id/}{CPMH}, \href{https://hpu.ugm.ac.id}{HPU UGM}, dan \href{https://repository.kemkes.go.id/}{Kementerian Kesehatan}.
	\item	\textbf{Variabel}: Penelitian ini menggunakan variabel bebas berupa metode \textit{knowledge extraction} dan \textit{knowledge retrieval} yang diterapkan.
	      Variabel terikat adalah performa \textit{chatbot} yang berupa kualitas respons yang dinilai berdasarkan metrik tertentu.
	\item	\textbf{Hipotesis}: Hipotesis yang diusulkan dalam penelitian ini adalah \textit{chatbot} kesehatan mental yang menggunakan RAG berbasis \textit{Knowledge Graph} dengan optimalisasi pada \textit{knowledge extraction} dan \textit{knowledge retrieval} menunjukkan peningkatan performa dibandingkan dengan \textit{chatbot} yang menggunakan pendekatan RAG standar atau tanpa \textit{Knowledge Graph}.
	\item	\textbf{Keterbatasan Penelitian}: Penelitian ini memiliki keterbatasan pada \textit{dataset} yang digunakan sebagai basis pengetahuan dalam sistem RAG.
	      Akurasi dan kelengkapan data bergantung pada \textit{dataset} yang mungkin belum mencakup semua aspek permasalahan kesehatan mental.
	      Penelitian ini tidak mengevaluasi dampak klinis \textit{chatbot} pada kondisi psikologis pengguna melainkan hanya mengevaluasi hasil respons \textit{chatbot} dengan metrik yang tersedia.
\end{enumerate}

\section{Manfaat Penelitian}
Penelitian ini diharapkan memberikan kontribusi pada pengembangan arsitektur \textit{chatbot} yang lebih andal dan terpercaya, khususnya dalam domain yang memerlukan informasi berbasis bukti seperti layanan kesehatan mental.
Selain itu, penelitian ini juga diharapkan memberikan wawasan mengenai efektivitas arsitektur \textit{chatbot} menggunakan RAG berbasis \textit{Knowledge Graph} dalam memberikan respons yang akurat.
Dari sisi praktis, \textit{chatbot} ini dapat diimplementasikan dan memberikan dampak positif dengan membantu pengguna dalam membuat keputusan terkait dengan kesejahteraan mental mereka, dengan tetap menekankan pentingnya konsultasi profesional untuk diagnosis dan perawatan.
\section{Sistematika Penulisan}
Sistematika penulisan berisi hal yang akan dibahas dalam penelitian ini yang akan disusun menjadi 5 bab, sebagai berikut,

\begin{enumerate}
	\item \textbf{Bab I Pendahuluan} menjelaskan hal yang melatarbelakangi adanya penelitian ini, rumusan masalah yang akan coba dicari solusinya, tujuan yang ingin dicapai, serta manfaat bagi pihak-pihak terkait.
	\item \textbf{Bab II Tinjauan Pustaka dan Dasar Teori} berisi ulasan literatur penelitian yang sudah pernah dilakukan sebelumnya mengenai topik-topik yang relevan dengan pengembangan chatbot.
	      Bab ini akan menguraikan beberapa metode yang pernah dilakukan untuk meningkatkan performa \textit{chatbot} pada aplikasi tertentu khususnya pada bidang layanan kesehatan mental.
	      Setiap metode tersebut akan diulas potensi untuk pengembangan lebih lanjut, peningkatan yang telah dicapai, dan tantangan yang mungkin dihadapi.
	      Selain itu, bab ini juga akan disertai dengan penjelasan mengenai teori-teori yang berkaitan dengan pengembangan \textit{chatbot}, arsitektur RAG, dan penggunaan \textit{Knowledge Graph}.
	\item \textbf{Bab III Metode Penelitian} membahas langkah-langkah detail yang akan ditempuh dalam penelitian seperti mulai dari identifikasi masalah, studi literatur, proses pengumpulan data, perancangan dan pengembangan model, hingga metode evaluasi dan analisis hasil.
	      Bab ini juga akan menjelaskan metode penelitian eksperimen dan kuantitatif yang akan dilakukan dalam penelitian ini.
	\item \textbf{Bab IV Hasil dan Pembahasan} menyajikan hasil dari implementasi dan pengujian model \textit{chatbot} yang telah dikembangkan.
	      Pada bab ini akan membahas mengenai kualitas respons dari \textit{chatbot} yang mengimplementasikan arsitektur RAG berbasis \textit{Knowledge Graph} dan dibandingkan dengan yang tidak menggunakan arsitektur tersebut.
	      Hasil temuan tersebut akan dianalisis secara mendalam dan dibandingkan dengan hipotesis serta penelitian sebelumnya.
	      \vspace{5cm}
	\item \textbf{Bab V Kesimpulan dan Saran}  berisi kesimpulan yang ditarik dari keseluruhan hasil penelitian dan pembahasan.
	      Bagian ini akan memberikan gambaran mengenai kontribusi penelitian ini dalam pengembangan teknologi \textit{chatbot} pada layanan kesehatan mental.
	      Selain itu, akan disampaikan pula saran-saran untuk pengembangan lebih lanjut dari sistem yang dibangun atau untuk penelitian terkait di masa mendatang
\end{enumerate}