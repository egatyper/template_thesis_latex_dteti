\textit{
	The use of chatbots as an accessible facilities for mental health support faces a crucial challenge regarding information accuracy.
	Common chatbot architectures, such as the conventional Retrieval-Augmented Generation (RAG) approach (Naive RAG) which relies on document chunks, carry a risk of generating misinformation and hallucinations due to a lack of comprehensive contextual understanding.
	This research aims to address this issue by designing, implementing, and evaluating a RAG architecture enriched with a structured knowledge base in the form of a Knowledge Graph (KG).
	The proposed methodology focuses on three main pillars that is the design of a RAG architecture that integrates a KG to ensure consistency and accuracy, the optimization of the Knowledge Extraction (KE) process using an LLM to build the KG from mental health literature, and the evaluation of various Knowledge Retrieval (KR) methods to find the most relevant information.
	Experimental results show that the RAG-KG architecture significantly outperforms the Naive RAG approach, demonstrated by an 8\% increase in correctness, a 4\% increase in faithfulness, and a 17\% increase in relevancy.
	This study confirms that the use of a Knowledge Graph, built and accessed through an optimized method, is a robust approach to enhance the reliability and security of chatbots.
	The result is a system capable of delivering more factual, relevant, and evidence-based mental health information, while minimizing the risk of hallucinations.
}

\noindent\textbf{Keywords} : RAG, knowledge graph, knowledge extraction, knowledge retrieval, chatbot, mental health, LLM.