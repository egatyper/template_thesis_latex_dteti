\textit{
	The use of chatbots as accessible mental health support tools faces a crucial challenge: the inherent risk of misinformation and "hallucination" in Large Language Models (LLMs).
	This research aims to address this problem by designing, implementing, and evaluating a Retrieval-Augmented Generation (RAG) architecture built upon a structured Knowledge Graph (KG).
	The proposed methodology focuses on three main pillars that is designing a RAG architecture that integrates the KG to ensure consistency and accuracy, optimizing the Knowledge Extraction process using an LLM to build the KG from mental health literature, and evaluating various Knowledge Retrieval methods to find the most relevant information.
	Experimental results show that the RAG-KG architecture significantly outperforms the Naive RAG approach, which is based on document chunks in almost all evaluation metrics.
	In the evaluation of retrieval methods, the Neighbor Expansion strategy proved to be the most effective overall, surpassing pure vector-based search methods in generating high-quality final answers.
	This research confirms that the use of a Knowledge Graph, built and accessed via optimized methods, is a robust approach to enhancing chatbot reliability and safety.
	The result is a system capable of delivering more factual, relevant, and evidence-based mental health information, while simultaneously minimizing the risk of hallucination.
}

\noindent\textbf{Keywords} : RAG, knowledge graph, knowledge extraction, knowledge retrieval, chatbot, mental health, LLM.