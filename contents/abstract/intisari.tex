Pemanfaatan \textit{chatbot} sebagai sarana pendukung kesehatan mental yang aksesibel dihadapkan pada tantangan krusial terkait akurasi informasi.
Arsitektur \textit{chatbot} yang umum digunakan, seperti pendekatan \textit{Retrieval-Augmented Generation} (RAG) berbasis potongan dokumen (Naive RAG) memiliki risiko dalam menghasilkan misinformasi dan halusinasi akibat kurangnya pemahaman kontekstual secara komprehensif.
Penelitian ini bertujuan untuk mengatasi masalah tersebut dengan merancang, mengimplementasikan, dan mengevaluasi sebuah arsitektur RAG yang diperkaya dengan basis pengetahuan terstruktur berupa \textit{Knowledge Graph} (KG).
Metodologi yang diusulkan berfokus pada tiga pilar utama yaitu perancangan arsitektur RAG yang mengintegrasikan KG untuk menjamin konsistensi dan akurasi, optimalisasi proses \textit{Knowledge Extraction} (KE) menggunakan LLM untuk membangun KG dari literatur kesehatan mental, dan evaluasi berbagai metode \textit{Knowledge Retrieval} (KR) untuk menemukan informasi yang paling relevan.
Hasil pengujian menunjukkan bahwa arsitektur RAG-KG secara signifikan mengungguli pendekatan Naive RAG berbasis potongan dokumen pada hampir semua metrik uji yang ditunjukkan dengan peningkatan \textit{correctness} sebesar 8\% \textit{faithfulness} sebesar 4\%  dan \textit{relevancy} sebesar 17\%.
Penelitian ini mengonfirmasi bahwa penggunaan \textit{Knowledge Graph} yang dibangun dan diakses melalui metode yang teroptimalkan merupakan pendekatan yang kuat untuk meningkatkan keandalan dan keamanan \textit{chatbot}.
Hasilnya adalah sebuah sistem yang mampu menyajikan informasi kesehatan mental yang lebih faktual, relevan, dan berbasis bukti, sekaligus meminimalkan risiko halusinasi.

\noindent{Kata kunci} : RAG, \textit{knowledge graph}, \textit{knowledge extraction}, \textit{knowledge retrieval}, \textit{chatbot}, kesehatan mental, LLM.

% \vspace{1cm}

% %HAPUS YANG TIDAK PERLU
% %-------------------------------------------------
% \noindent\fbox{%
% 	\parbox{\textwidth}{%
% \textbf{Contoh Abstrak Teknik Elektro:} \\

% \hspace{1cm} "Penelitian ini bertujuan untuk mengembangkan sistem pengendalian suhu ruangan dengan menggunakan microcontroller. Metodologi yang digunakan adalah desain sirkuit, implementasi sistem pengendalian, dan pengujian performa. Hasil penelitian menunjukkan 
% bahwa sistem pengendalian suhu ruangan yang dikembangkan mampu mengendalikan suhu ruangan dengan akurasi sebesar ±0,5°C. Kesimpulan dari penelitian ini adalah sistem pengendalian suhu ruangan yang dikembangkan efektif dan efisien. \\

% Kata kunci: microcontroller, sistem pengendalian suhu, akurasi."
% \vspace{5mm}

% \textbf{Contoh Abstrak Teknik Biomedis:} \\

% \hspace{1cm} "Penelitian ini bertujuan untuk mengevaluasi keefektifan prototipe alat pemantau denyut jantung berbasis elektrokardiogram (ECG) untuk pasien jantung. Metodologi yang digunakan meliputi desain dan pembuatan prototipe, pengujian dengan pasien, dan analisis data. Hasil penelitian menunjukkan bahwa prototipe alat pemantau denyut jantung berbasis ECG memiliki 
% akurasi yang baik dan mampu memantau denyut jantung pasien secara efektif. Kesimpulan dari penelitian ini adalah prototipe alat pemantau denyut jantung berbasis ECG merupakan solusi 
% yang efektif dan efisien untuk memantau pasien jantung. \\

% Kata kunci: elektrokardiogram, alat pemantau denyut jantung, akurasi."
% \vspace{5mm}

% 	}%
% }

% %-------------------------------------------------

% \noindent\fbox{%
% 	\parbox{\textwidth}{%
% 		\textbf{Contoh Abstrak Teknologi Informasi:} \\

% \hspace{1cm} "Penelitian ini bertujuan untuk mengevaluasi keamanan dan privasi pengguna aplikasi media sosial terpopuler. Metodologi yang digunakan meliputi analisis kebijakan privasi dan pengaturan keamanan, pengujian penetrasi, dan survei pengguna. Hasil penelitian 
% menunjukkan bahwa beberapa aplikasi media sosial memiliki kebijakan privasi yang kurang jelas dan rendahnya tingkat keamanan. Kesimpulan dari penelitian ini adalah pentingnya meningkatkan kebijakan privasi dan tingkat keamanan pada aplikasi media sosial untuk melindungi privasi dan data pengguna. \\

% Kata kunci: media sosial, keamanan, privasi, pengguna."
% \vspace{5mm}

% 	}%
% }