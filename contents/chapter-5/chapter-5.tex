\chapter{Kesimpulan dan Saran}
\section{Kesimpulan}
Berdasarkan penelitian yang telah dilakukan, metode RAG berbasis \textit{Knowledge Graph} secara umum mengungguli Naive RAG yang berbasis potongan dokumen pada mayoritas metrik uji.
Hasil evaluasi terhadap beberapa metode pengambilan konteks dari \textit{Knowledge Graph} serta analisis kualitas jawaban akhir menghasilkan beberapa kesimpulan utama sebagai berikut:

\begin{enumerate}
      \item Arsitektur RAG Berbasis \textit{Knowledge Graph} terbukti mampu meningkatkan akurasi dan konsistensi respons.
            Penelitian ini berhasil merancang dan mengimplementasikan arsitektur \textit{chatbot} yang mengintegrasikan RAG dengan KG.
            Hasil evaluasi menunjukkan bahwa arsitektur ini secara konsisten mengungguli pendekatan Naive RAG, yang dibuktikan dengan peningkatan \textlatin{correctness} sebesar 8\%, \textit{relevancy} sebesar 17\%, dan \textit{faithfulness} sebesar 4\%.
            Keunggulan tersebut menegaskan bahwa arsitektur yang diusulkan mampu menghasilkan respons yang lebih akurat secara faktual, relevan, dan konsisten dengan basis pengetahuannya, sehingga meminimalkan masalah fundamental terkait risiko misinformasi.

      \item Metode \textit{Knowledge Extraction} berhasil mengonstruksi KG yang efektif untuk proses \textit{retrieval}.
            Struktur relasi yang terbentuk melalui proses ekstraksi terbukti kaya dan terorganisir, sehingga memungkinkan proses penelusuran informasi yang lebih menyeluruh.
            Hal ini tercermin dari performa \textit{retrieval} yang unggul pada strategi berbasis penelusuran graf, menunjukkan bahwa fondasi pengetahuan yang dibangun cukup kuat untuk mendukung pemanggilan informasi secara komprehensif.

      \item Strategi \textit{Neighbor Expansion} menjadi pendekatan \textit{retrieval} paling efektif.
            Dari berbagai strategi yang diuji, metode \textit{Neighbor Expansion} secara konsisten menghasilkan kualitas jawaban terbaik meskipun terdapat \textit{trade-off} antara presisi dan \textit{recall}.
            Pendekatan ini mampu menjangkau informasi yang lebih luas tanpa mengorbankan relevansi jawaban akhir, sekaligus menegaskan keunggulan penelusuran berbasis struktur graf dibandingkan pencarian berbasis vektor murni.
\end{enumerate}

Secara keseluruhan, penelitian ini menegaskan bahwa penggunaan \textit{Knowledge Graph} sebagai fondasi pengetahuan dalam RAG dapat meningkatkan relevansi, akurasi, dan faktualitas respons \textit{chatbot}.
Dengan memanfaatkan sumber data berupa buku dan artikel terkait kesehatan mental, sistem yang dikembangkan mampu menyajikan jawaban yang lebih informatif dan faktual, sehingga berpotensi meminimalkan risiko halusinasi saat berinteraksi dengan pengguna.

\section{Saran}
Berdasarkan penelitian yang telah dilakukan, berikut adalah beberapa saran untuk penelitian dan pengembangan selanjutnya:

\begin{enumerate}
      \item Pada penelitian ini \textit{dataset} evaluasi masih diperoleh berupa data sintetis hasil \textit{prompting} menggunakan LLM.
            Penelitian selanjutnya disarankan menggunakan \textit{dataset} pertanyaan dan jawaban yang telah divalidasi oleh ahli atau bersumber langsung dari sumber terpercaya agar evaluasi mencerminkan performa saat digunakan dalam kasus yang sebenarnya.

      \item Mengingat belum adanya standar baku untuk struktur \textit{Knowledge Graph} dan cara mengevaluasinya, penelitian mendatang dapat berfokus pada evaluasi dampak dari berbagai desain KG terhadap kualitas jawaban akhir.
            Evaluasi ini akan melengkapi fokus penelitian saat ini yang lebih terpusat pada metrik \textit{retrieval} dan kualitas jawaban.

      \item Perlu dilakukan eksplorasi lebih lanjut terhadap \textit{tuning} parameter, seperti nilai top-k dan \textit{max hop} pada proses \textit{retrieval}, untuk menemukan konfigurasi optimal yang dapat menyeimbangkan antara kelengkapan dan relevansi konteks yang diambil.

      \item Meskipun berkinerja kurang baik, konsep pencarian jalur seperti pada \textit{N-Shortest Path} tetap memiliki potensi.
            Penelitian lebih lanjut dapat difokuskan pada penerapan mekanisme penyaringan atau pembobotan untuk memastikan hanya jalur yang paling relevan secara semantik dan kontekstual yang dipertimbangkan, sehingga dapat mengurangi risiko halusinasi.

\end{enumerate}